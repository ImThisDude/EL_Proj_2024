%
% 6.006 problem set 0 solutions template
%
\documentclass[11pt,twoside]{article}

\input{macros-sp20}
\newcommand{\theproblemsetnum}{3}
\usepackage{xcolor}
\usepackage{caption}
\usepackage{geometry}
\usepackage{hyperref}
\usepackage{float}
\usepackage{booktabs} % For better horizontal rules
\usepackage{tabularx} % For flexibility in column widths
\usepackage{multirow}
\usepackage{array}
\usepackage{changepage} % for the adjustwidth environment
\usepackage{lipsum}
\usepackage{tikz}
\usepackage{adjustbox}
\usetikzlibrary{shapes, arrows.meta, positioning}
\usepackage{mathtools}
\usepackage{amsmath}
\usepackage{subcaption}

\newcommand{\labels}[2]{%
\\ \textcolor{cyan}{$#1+$}, \textcolor{magenta}{$#2-$}%
}

\setlength{\parskip}{0.5em} % Adjust to desired spacing

\title{UT ML Problem Set 2}

\begin{document}
\handout{Project Report}

\setlength{\parindent}{0pt}

\begin{center}

{\LARGE \textbf{An Overview of Language Models }} \\[20pt]

\textbf{ --- Fall 2024 --- } \\[15pt]

\medskip

{\large \textbf{Project Collaborators:}}
\medskip

\begin{table}[h]
\centering
\begin{tabularx}{0.8\textwidth}{>{\centering\arraybackslash}X >{\centering\arraybackslash}X}
\toprule
\textbf{Name} & \textbf{Student ID} \\
\midrule
Mobin Roohi & 610300060 \\
Amirhossein Ghorbaninezhad &  610xxxxxx\\
Vida Karbasi & 610xxxxxx \\
Ali Ghozati & 610xxxxxx \\
\bottomrule \\[20pt]
\end{tabularx}
\end{table}
\end{center}


\begin{abstract}
\textit{}

\textit{
}

\textit{
}

\textit{
}
\end{abstract}

\newpage

%%%%%%%%%%%%%%%%%%%%%%%%%%%%%%%%%%%%%%%%%%%%%%%%%%%%%
% See below for common and useful latex constructs. %
%%%%%%%%%%%%%%%%%%%%%%%%%%%%%%%%%%%%%%%%%%%%%%%%%%%%%

% Some useful commands:
% $f(x) = \Theta(x)$
% $T(x, y) \leq \log(x) + 2^y + \binom{2n}{n}$
% \ttt{code\_function}


% You can create unnumbered lists as follows:
% \begin{itemize}
%     \item First item in a list
%         \begin{itemize}
%             \item First item in a list
%                 \begin{itemize}
%                     \item First item in a list
%                     \item Second item in a list
%                 \end{itemize}
%             \item Second item in a list
%         \end{itemize}
%     \item Second item in a list
% \end{itemize}

% You can create numbered lists as follows:
% \begin{enumerate}
%     \item First item in a list
%     \item Second item in a list
%     \item Third item in a list
% \end{enumerate}

% You can write aligned equations as follows:
% \begin{align}
%     \begin{split}
%         (x+y)^3 &= (x+y)^2(x+y) \\
%                 &= (x^2+2xy+y^2)(x+y) \\
%                 &= (x^3+2x^2y+xy^2) + (x^2y+2xy^2+y^3) \\
%                 &= x^3+3x^2y+3xy^2+y^3
%     \end{split}
% \end{align}

% You can create grids/matrices as follows:
% \begin{align}
%     A =
%     \begin{bmatrix}
%         A_{11} & A_{21} \\
%         A_{21} & A_{22}
%     \end{bmatrix}
% \end{align}


%‌ PCA‌ %%%%%%%%%%%%%%%%%%%%%%%%%%%%%%%%%%%%%%%%%%%%%%%%%%

{
\setlength{\parskip}{0.35em} 
\tableofcontents
}
\newpage



\addcontentsline{toc}{section}{Bibliography}
\bibliographystyle{unsrt}
\bibliography{Biby.bib}


\end{document}
