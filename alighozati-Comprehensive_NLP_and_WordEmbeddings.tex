
\maketitle

\tableofcontents
\newpage

\section{Introduction to Natural Language Processing (NLP)}
Natural Language Processing (NLP) focuses on the interaction between computers and human language. Its applications span a wide range of domains, including search engines, virtual assistants, and social media analysis.

\subsection{Defining NLP}
NLP is the field of artificial intelligence that equips machines with the ability to read, interpret, and generate human language. It bridges linguistics and computer science to enable:
\begin{itemize}
    \item Machine translation.
    \item Text summarization.
    \item Sentiment analysis.
    \item Speech recognition.
\end{itemize}

\subsection{Key Challenges}
\begin{enumerate}
    \item \textbf{Ambiguity:} Words or sentences can have multiple interpretations.
    \item \textbf{Data Scarcity:} Many languages and domains lack annotated datasets.
    \item \textbf{Dynamic Context:} Meaning can shift depending on context, requiring models to adapt.
\end{enumerate}

\subsection{Applications of NLP}
\begin{itemize}
    \item \textbf{Search and Retrieval:} Google uses NLP to improve query results.
    \item \textbf{Chatbots:} Virtual assistants like Alexa and Siri rely on NLP.
    \item \textbf{Content Moderation:} Automatically flagging harmful or inappropriate content.
    \item \textbf{Machine Translation:} Services like Google Translate facilitate communication across languages.
\end{itemize}

\section{Traditional Approaches in NLP}
Before the advent of deep learning, NLP relied heavily on statistical models and hand-engineered features.

\subsection{N-grams for Language Modeling}
An \(n\)-gram is a sequence of \(n\) words used to estimate the likelihood of word sequences:
\[
P(w_1, w_2, \ldots, w_n) = \prod_{i=1}^{n} P(w_i \mid w_{i-1}, w_{i-2}, \ldots, w_{i-n+1}).
\]
For example, in trigrams (\(n=3\)):
\[
P(\text{sentence}) = P(w_1) P(w_2 \mid w_1) P(w_3 \mid w_1, w_2).
\]

\subsection{Bag-of-Words (BoW)}
BoW represents text by counting word frequencies, disregarding order:
\[
\text{Document: "cats chase mice"} \to [1, 0, 1, 1].
\]
While simple, this method loses syntactic relationships between words.

\subsection{One-Hot Encoding}
Each word in a vocabulary is represented by a binary vector:
\[
\text{Vocabulary: ["cat", "dog", "fish"]}, \quad \text{cat} \to [1, 0, 0].
\]
This approach leads to sparsity and lacks semantic relationships.

\section{Deep Learning for NLP}
Deep learning revolutionized NLP by introducing models capable of automatically learning features from data.

\subsection{Neural Networks Overview}
Neural networks consist of interconnected layers:
\begin{itemize}
    \item \textbf{Input Layer:} Processes raw data (e.g., word embeddings).
    \item \textbf{Hidden Layers:} Transform inputs through weights and activation functions.
    \item \textbf{Output Layer:} Produces predictions (e.g., next word in a sequence).
\end{itemize}

\subsection{Recurrent Neural Networks (RNNs)}
RNNs model sequential data by maintaining hidden states across time:
\[
h_t = f(W_{hh} h_{t-1} + W_{xh} x_t),
\]
where \(W_{hh}\) and \(W_{xh}\) are weight matrices. Variants include:
\begin{itemize}
    \item \textbf{LSTM:} Introduces gates to control information flow and address long-term dependencies.
    \item \textbf{GRU:} Simplifies the LSTM architecture while retaining performance.
\end{itemize}

\subsection{Applications of Neural Networks in NLP}
\begin{itemize}
    \item \textbf{Text Classification:} Labeling documents as spam or non-spam.
    \item \textbf{Machine Translation:} Converting text between languages.
    \item \textbf{Sentiment Analysis:} Extracting emotion from text.
\end{itemize}

\section{Word Embeddings}
Word embeddings map words to dense, low-dimensional vectors in a continuous space.

\subsection{Motivation and Concept}
Word embeddings capture semantic meaning:
\[
\text{king} - \text{man} + \text{woman} = \text{queen}.
\]
This enables models to generalize across tasks, unlike one-hot encoding.

\subsection{Training Techniques}
\subsubsection{Word2Vec}
Word2Vec employs two architectures:
\begin{itemize}
    \item \textbf{CBOW:} Predicts a word given its context.
    \item \textbf{Skip-Gram:} Predicts context words given a target word.
\end{itemize}

\subsubsection{GloVe}
GloVe captures global co-occurrence statistics to produce embeddings that reflect word relationships.

\subsubsection{FastText}
FastText represents words as character n-grams, handling out-of-vocabulary words effectively.

\subsection{Advanced Embedding Models}
\begin{itemize}
    \item \textbf{ELMo:} Context-sensitive embeddings derived from a bidirectional language model.
    \item \textbf{BERT:} Generates contextualized embeddings by processing text bidirectionally.
    \item \textbf{Transformers:} Architectures like GPT enable dynamic context handling.
\end{itemize}

\subsection{Applications of Word Embeddings}
\begin{itemize}
    \item \textbf{Semantic Similarity:} Identifying related words.
    \item \textbf{Named Entity Recognition (NER):} Extracting entities like names or dates.
    \item \textbf{Text Summarization:} Generating concise text summaries.
\end{itemize}

\subsection{Limitations of Word Embeddings}
\begin{itemize}
    \item \textbf{Biases:} Gender and societal biases can propagate into embeddings.
    \item \textbf{Context Insensitivity:} Static embeddings fail to capture varying meanings of words.
\end{itemize}

\section{Mathematical Foundations of Word Embeddings}
\subsection{Distributional Semantics}
The distributional hypothesis states:
\[
\text{"Words used in similar contexts have similar meanings."}
\]
This underpins the creation of word embeddings.

\subsection{Objective Functions}
Word2Vec minimizes the negative log-likelihood of predicting context words:
\[
J = -\sum_{w \in C} \log P(w \mid t),
\]
where \(P(w \mid t)\) is the probability of a context word \(w\) given target \(t\).

\subsection{Similarity Metrics}
Common metrics for comparing word embeddings:
\begin{itemize}
    \item \textbf{Cosine Similarity:}
    \[
    \text{sim}(u, v) = \frac{u \cdot v}{\|u\| \|v\|}.
    \]
    \item \textbf{Euclidean Distance:}
    \[
    d(u, v) = \sqrt{\sum_{i=1}^n (u_i - v_i)^2}.
    \]
\end{itemize}

\section{Future Directions in NLP}
\subsection{Bias Mitigation}
Developing debiasing methods for word embeddings to reduce societal stereotypes.

\subsection{Multilingual Models}
Improving support for low-resource languages and multilingual applications.

\subsection{Dynamic Representations}
Enhancing embeddings to adapt to varying contexts dynamically.

\textbf{References:}
\begin{itemize}
    \item Tomas Mikolov, \textit{Neural Networks for Natural Language Processing Slides}, 2017.
    \item Danna Gurari, \textit{Introduction to NLP and Word Embeddings}, 2022.
\end{itemize}

